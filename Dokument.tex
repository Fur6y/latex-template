% Allgemeine Dokument Einstellungen
\documentclass[
	12pt,				% Schriftgröße (12pt)
	a4paper,			% Papierformat (a4)
	oneside,			% Drucklayout (einseitig bedruckt)
	ngerman,			% Sprache (deutsch, neue Rechtschreibung), globale Sprachoption für "babel"
	titlepage,			% Titelblatt (ja)
	bibliography=totoc  % fügt das Literaturverzeichnis ins Inhaltsverzeichnis ein
	%draft				% DEBUG, markiert fehlerhafte Zeilen rot
]{scrartcl}				% Koma-Script Dokumenttyp "article"

\KOMAoptions{
	twoside=false,
	headinclude=true,
	footinclude=true,
	DIV=calc % legt Wert zur Blattaufteilung bei der Berechnung des Satzspiegels fest
}

% Sprache, Silbentrennung
\usepackage[ngerman]{babel}
\usepackage[babel,german=guillemets]{csquotes}

% westeuropäischer Zeichenkodierung
\usepackage[T1]{fontenc}
% Zeichenkodierung UTF-8
\usepackage[utf8]{inputenc}
% Latin Modern Font (Nachfolger von Computer Modern)
\usepackage{lmodern}

% Microtypografie Anpassungen
\usepackage{microtype}

% ermöglicht das Ausgeben eines Kapiteltitels
\usepackage{titleref}

% Literaturverzeichnis mit Biblatex
\usepackage[
	backend=bibtex,
	bibencoding=utf8,
	sortcites=true,
	style=alphabetic,
	natbib=true,
	block=space,
	url=true,
	language=german
]{biblatex}
% Quellen.bib einbinden
\addbibresource{Quellen.bib}

% um Grafiken einbinden zu können
\usepackage[final]{graphicx}

% textumflossene Objekte, wie z.B. Bilder
\usepackage{wrapfig}

% zum Anlegen von Abkürzungen / Abkürzungsverzeichnis
% Optionen: printonlyused, withpage
\usepackage{acronym}

% legt Variablen für die Dokumentdaten (Titel, Author und Datum) an
\usepackage{titling}

% Paket für Textfarben
\usepackage{xcolor}

% Mehrspaltig
\usepackage{multicol}

% Seitennummerierung (roman, arabic)
\pagenumbering{arabic}

% Zeilenabstand auf 1,5fach
\linespread{1.5}

% Einrücktiefe der ersten Zeile eines Absatzes
\setlength{\parindent}{0pt}

% verhindert eine einzelne Zeile am Anfang des Absatzes (Schusterjungen)
\clubpenalty = 10000
% verhindert eine einzelne Zeile am Ende des Absatzes (Hurenkinder)
\widowpenalty = 10000
\displaywidowpenalty = 10000

% Packet für Seitenrandabständex und Einstellung für Seitenränder
\usepackage{geometry}
\geometry{
	a4paper,
	includehead,
	left=2.5cm,
	right=3cm,
	top=2.5cm,
	bottom=3cm,			% Abstand unten (bis zum Text)
	headheight=10pt,	% Höhe der Kopfzeile
	headsep=2.5em, 		% Abstand zwischen Kopfzeile und Text
	footskip=1cm,		% Höhe der Fusszeile
	voffset=0mm,
	hoffset=0mm
}
% Packet zum Ausgeben der aktuellen Layouteinstellungen
\usepackage{layout}

% Packet für URLs
\usepackage{url}

% Typografisch korrekte Abkürzung von "z.B."
\usepackage{xspace}
\newcommand{\zB}{\mbox{z.\,B.}\xspace}

% Daten
\title{Titel}
\newcommand{\thesubtitle}{Untertitel}
\subtitle{\thesubtitle}
\author{Vorname Nachname}
\date{Januar 2016}

% Variablen
\newcommand{\titleDocumentType}{Modul}
\newcommand{\degree}{BSc. Studiengangbezeichnung}
% Footer Text
\newcommand{\footer}{\titleDocumentType, HSD, FB Medien, \degree{}, \theauthor, \thedate}

% Koma-Script Kopf- und Fusszeilen Styling
\usepackage[draft=false]{scrlayer-scrpage}
\setkomafont{pagehead}{\footnotesize}
\setkomafont{pagefoot}{\footnotesize}
\ihead{\normalfont \textcolor{darkgray}{\thetitle}} % Kopfzeile links
\chead{} % Kopfzeile zentriert
\ohead{\normalfont \textcolor{darkgray}{Seite \normalsize\thepage}} % Kopfzeile rechts
\ifoot{\normalfont \textcolor{darkgray}{\footer}} % Fusszeile links
\cfoot{} % Fusszeile zentriert
\ofoot{} % Fusszeile rechts




\begin{document}

\begin{titlepage}
\begin{flushright}

\includegraphics{img/hsd_logo}

\vspace{\baselineskip}
\vspace{\baselineskip}


{\huge\textbf{\thetitle}\par}

{\Large\textbf{\thesubtitle}\par}

\vspace{\baselineskip}

{\LARGE\textbf{\theauthor}\par}
{\LARGE\textbf{000000}\par}


% restlichen Abstand auffüllen
\vfill


{\titleDocumentType} im Studiengang

\textbf{\degree}

\textbf{\thedate}

\vspace{\baselineskip}
\vspace{\baselineskip}

Betreuer:

Prof. Name

\end{flushright}
\end{titlepage}

\newpage
\section*{Eidesstattliche Erklärung}

Hiermit versichere ich an Eides statt, dass ich diese Arbeit eigenständig verfasst und keine anderen als die angegebenen und bei Zitaten kenntlich gemachten Quellen und Hilfsmittel benutzt habe.

\vspace{\baselineskip}
\vspace{\baselineskip}
\vspace{\baselineskip}

\line(1,0){250}

Datum, Unterschrift

\vfill

\textbf{Kontaktinformationen}

\bigskip

\theauthor

Strasse 0

00000 Ort

\medskip

E-Mail: mail@addresse.de

\vspace{\baselineskip}
\vspace{\baselineskip}
\vspace{\baselineskip}

\include{_Zusammenfassung}

\tableofcontents

\include{_Einleitung}

\include{_Inhalt}

\section{Fazit und Ausblick}

Fazit

\section*{Abkürzungsverzeichnis}
% Abkürzungsverzeichnis soll im Inhaltsverzeichnis auftauchen
\addcontentsline{toc}{section}{Abkürzungsverzeichnis}

\begin{acronym}[----------------] % längste Abkürzung steht in eckigen Klammern
	% A
	% B
	% C
	% D
	% E
	% F
	% G
	% H
	% I
	% J
	% K
	% L
	% M
	% N
	% O
	% P
	% Q
	% R
	% S
	% T
	% U
	% V
	% W
	% X
	% Y
	% Z
\end{acronym}


% zeige alle Quellen an, egal ob diese verwendet wurden oder nicht
\nocite{*}

\printbibliography[
	title={Literaturverzeichnis}
]

\newpage

\appendix

\addcontentsline{toc}{section}{Anhang}
\section*{Anhang}

Anhang

\section{Anhang}

Text

\subsection{Anhang}

Text

\end{document}